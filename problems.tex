\documentclass{article}
\usepackage[utf8]{inputenc}
\usepackage[T2A]{fontenc}
\usepackage[bulgarian]{babel}

\title{Курсова работа по Времеви редове}
\author{Йоана Левчева, ф.н. 26527}

\usepackage{natbib}
\usepackage{graphicx}

\begin{document}

\maketitle

\section*{Задача 1}
\textbf{Отговорете на следните въпроси:}
\begin{flushleft}

\textbf{1. Какво представлява времевият ред?}
\begin{flushleft}
Времеви ред наричаме множество от наблюдения $\{ x_t \}$, всяко едно от които е регистрирано в определен момент от време $t$.
\end{flushleft}

\textbf{2. Опишете компонентите на времевия ред.}
\begin{flushleft}
Можем да представим процеса $\{X_t \}$ като реализация на процеса $X_t = m_t + s_t + Y_t$, където $m_t$ е компонента на тренда (бавно изменяща се функция), $s_t$ е функция с известен период $d$, която наричаме сезонна компонента и $Y_t$ е шумова компонента, която е стационарна.
\end{flushleft}

\textbf{3. Напишете формулата за автокорелация и обяснете какво измерва тя.}
\begin{flushleft}
Нека $X_t$ да е стационарен времеви ред. Автокорелационна формула на $\{X_t \}$ наричаме $$\rho_{X}(h) = \frac{\gamma_{X}(h)}{\gamma_{X}(0)} = Corr(X_{t+h}, X_t) = \frac{Cov(X_{t+h}, X_t)}{\sqrt{Var X_{t+h}Var X_t}},$$ където $\gamma_{X}(h) = Cov(X_{t+h}, X_t)$ е автоковариационната функция на $\{ X_t \}$. Автокорелацията представлява корелацията между величината и нейното закъснение за един или повече периода от време.
\end{flushleft}

\textbf{4. Какво е корелограма и за какво се използва тя?}
\begin{flushleft}
Корелограмата представлява графика на коефициентите на автокорелация за различни периоди на закъснение за дадения времеви ред. Показва изменението на автокорелацията с течение на времето. 
\end{flushleft}

\textbf{5. Всяко от следващите твърдения описва стационарен или нестационарен ред. Определете към кой тип бихте причислили всеки от тях.}
\begin{flushleft}
(а) ред, имащ тренд $\rightarrow$ нестационарен \\
(б) ред, за който функцията на средните стойности е периодична функция \\
(в) ред, не съдържащ нито подеми, нито спадове \\
(г) Гаусов ред с нулево средно, чиито последователни стойности са некорелирани \\
(д) ред, чиято функция на средните стойности остава постоянна във времето, а амплитудата му нараства с течение на времето
\end{flushleft}

\end{flushleft}

\section*{Задача 2}
Нека да е даден следният времеви ред:


\end{document}
