\documentclass{article}
\usepackage[utf8]{inputenc}
\usepackage[T2A]{fontenc}
\usepackage[bulgarian]{babel}

\title{Курсова работа по Времеви редове}
\author{Йоана Левчева, ф.н. 26527}

\usepackage{natbib}
\usepackage{graphicx}

\begin{document}

\maketitle

\section*{Задача 1}
\textbf{Отговорете на следните въпроси:}
\begin{flushleft}

\textbf{1. Какво представлява времевият ред?}
\begin{flushleft}
Времеви ред наричаме множество от наблюдения $\{ x_t \}$, всяко едно от които е регистрирано в определен момент от време $t$.
\end{flushleft}

\textbf{2. Опишете компонентите на времевия ред.}
\begin{flushleft}
Можем да представим процеса $\{X_t \}$ като реализация на процеса $X_t = m_t + s_t + Y_t$, където $m_t$ е компонента на тренда (бавно изменяща се функция), $s_t$ е функция с известен период $d$, която наричаме сезонна компонента и $Y_t$ е шумова компонента, която е стационарна.
\end{flushleft}

\textbf{3. Напишете формулата за автокорелация и обяснете какво измерва тя.}
\begin{flushleft}
Нека $X_t$ да е стационарен времеви ред. Автокорелационна формула на $\{X_t \}$ наричаме $$\rho_{X}(h) = \frac{\gamma_{X}(h)}{\gamma_{X}(0)} = Corr(X_{t+h}, X_t) = \frac{Cov(X_{t+h}, X_t)}{\sqrt{Var X_{t+h}Var X_t}},$$ където $\gamma_{X}(h) = Cov(X_{t+h}, X_t)$ е автоковариационната функция на $\{ X_t \}$. Автокорелацията представлява корелацията между величината и нейното закъснение за един или повече периода от време.
\end{flushleft}

\textbf{4. Какво е корелограма и за какво се използва тя?}
\begin{flushleft}
Корелограмата представлява графика на коефициентите на автокорелация за различни периоди на закъснение за дадения времеви ред. Показва изменението на автокорелацията с течение на времето. 
\end{flushleft}

\textbf{5. Всяко от следващите твърдения описва стационарен или нестационарен ред. Определете към кой тип бихте причислили всеки от тях.}
\begin{flushleft}
(а) ред, имащ тренд $\rightarrow$ нестационарен \\
(б) ред, за който функцията на средните стойности е периодична функция $\rightarrow$ нестационарен \\
(в) ред, не съдържащ нито подеми, нито спадове $\rightarrow$ стационарен \\
(г) Гаусов ред с нулево средно, чиито последователни стойности са некорелирани $\rightarrow$ стационарен \\
(д) ред, чиято функция на средните стойности остава постоянна във времето, а амплитудата му нараства с течение на времето $\rightarrow$ нестационарен
\end{flushleft}

\textbf{6. По-долу са описани няколко типа времеви редове: случайни, стационарни, съдържащи тренд, сезонни. Определете типа на всеки един от тях.}
\begin{flushleft}
(а) ред, чиито основни статистически свойства като средно и дисперсия остават постоянну във времето $\rightarrow$ стационарен \\
(б) времеви ред, чиито последователни стойности не са свързани един с друг $\rightarrow$ случаен \\
(в) значителен автокорелационен коефициент се появява за лагове, кратни на 4 $\rightarrow$ сезонен \\
(г) коефициентите на корелация значително се различават от нула за първите няколко лага, а после постепенно бавно клонят към нула с увеличаването на лага $\rightarrow$ съдържащ тренд \\
(д) коефициентите на автокорелация значително се различават от нула за първите няколко лага, а после експоненциално намаляват към нула с увеличаването на лага $\rightarrow$ стационарен \\
(е) стойностите на реда имат постоянна тенденция към наратсване $\rightarrow$ съдържащ тренд
\end{flushleft}

\textbf{7. В кой метод на прогнозиране стойността на изучаваната величина в текущия период се счита за прогноза за следващия?}
\begin{flushleft}
При наивните методи се предполага, че последният период най-добре прогнозира бъдещето. Тъй като на текущата стойност на времевия ред се приписва стопроцентна тежест, може да наречем наивната прогноза "прогноза без изменение": $$\widehat{X}_{t+1} = X_t$$.
\end{flushleft}

\textbf{8. В кой наивен метод на прогнозиране на наблюденията се приписват равни теглови коефициенти?}
\begin{flushleft}
В метода на простите средни за създаване на прогноза за следващия период се използва средната стойност на всички минали стойности на времевия ред. 
\end{flushleft}

\textbf{9. Какъв наивен метод на прогнозиране трябва да изберем, ако данните имат сезонност?}
\begin{flushleft}
Наивен метод при данни със сезонност, ако сезонните вариации са достатъчни силни, може да бъде: $$\widehat{X}_{t+1} = X_{t-d+1},$$ където $d$ е сезонната компонента.
\end{flushleft}

\end{flushleft}

\section*{Задача 2}
\begin{flushleft}

\textbf{1. Каква прогноза за периода 9 ще получим по метода на плаващото средно за 5 периода?}
\begin{flushleft}
$\widehat{X}_9 = \frac{X_8 + X_7 + X_6 + X_5 + X_4}{5} = \frac{216 + 219 + 220 + 225 + 226}{5} = 221.2$
\end{flushleft}

\textbf{2. Експоненциално изглаждане при $\alpha = 0.2$:}
\begin{flushleft}
$\widehat{Y}_2 = \alpha Y_1 + (1 - \alpha) \widehat{Y}_1 = 0.2\times200 + 0.8\times200 = 200$ \\
$\widehat{Y}_3 = \alpha Y_2 + (1 - \alpha) \widehat{Y}_2 = 0.2\times210 + 0.8\times200 = 202$ \\
$\widehat{Y}_4 = \alpha Y_3 + (1 - \alpha) \widehat{Y}_3 = 0.2\times215 + 0.8\times202 = 204.6$ \\
Грешка от прогнозата за период 3: $$e_3 = Y_3 - \widehat{Y}_3 = 215 - 202 = 13.$$
За прогнозата за период 4 можем да очакваме отново по-ниска стойност от действителната, което наистина е така. \\
$MAD = \frac{1}{n} \sum_{t=1}^{n} |Y_t - \widehat{Y}_t| = \frac{1}{3} \sum_{t=1}^{3} |Y_t - \widehat{Y}_t|$ \\
$MSE = \frac{1}{n} \sum_{t=1}^{n} (Y_t - \widehat{Y}_t)^2 = \frac{1}{3} \sum_{t=1}^{3} (Y_t - \widehat{Y}_t)^2$ \\
$MAPE = \frac{1}{n} \sum_{t=1}^{n} \frac{|Y_t - \widehat{Y}_t|}{Y_t} = \frac{1}{3} \sum_{t=1}^{3} \frac{|Y_t - \widehat{Y}_t|}{Y_t}$ \\
$MAPE = \frac{1}{n} \sum_{t=1}^{n} \frac{(Y_t - \widehat{Y}_t)}{Y_t} = \frac{1}{3} \sum_{t=1}^{3} \frac{(Y_t - \widehat{Y}_t)}{Y_t}$ \\

$MAD = \frac{1}{n} \sum_{t=1}^{n} |Y_t - \widehat{Y}_t| = \frac{1}{4} \sum_{t=1}^{4} |Y_t - \widehat{Y}_t|$ \\
$MSE = \frac{1}{n} \sum_{t=1}^{n} (Y_t - \widehat{Y}_t)^2 = \frac{1}{4} \sum_{t=1}^{4} (Y_t - \widehat{Y}_t)^2$ \\
$MAPE = \frac{1}{n} \sum_{t=1}^{n} \frac{|Y_t - \widehat{Y}_t|}{Y_t} = \frac{1}{4} \sum_{t=1}^{4} \frac{|Y_t - \widehat{Y}_t|}{Y_t}$ \\
$MAPE = \frac{1}{n} \sum_{t=1}^{n} \frac{(Y_t - \widehat{Y}_t)}{Y_t} = \frac{1}{4} \sum_{t=1}^{4} \frac{(Y_t - \widehat{Y}_t)}{Y_t}$ \\
\end{flushleft}

\textbf{3. Избиране на оптимална стойност на $\alpha$:}
\begin{flushleft}
За оптимална оценка на $\alpha$ един от методите използва минимизация на $MSE$. Последователно се изчисляват прогнози за $\alpha$, равно на 0.1, 0.2, ..., 0.9 и се пресмята $MSE$ за всяка една от тях. Стойността на $\alpha$, за която $MSE$ се окаже най-малка, се избира за по-нататъшно използване в прогнозите.
\end{flushleft}

\end{flushleft}

\section*{Задача 3}
\begin{flushleft}

\textbf{1.} вярно \\
\textbf{2.} вярно \\
\textbf{3.} вярно \\
\textbf{4.} грешно

\end{flushleft}

\section*{Задача 4}
\begin{flushleft}

\textbf{1. $X_t = a + bZ_0$}
\begin{flushleft}
$EX_t = E(a + bZ_0) = E(a) + E(bZ_0) = a + bE(Z_0) = a,$ което не зависи от $t$. \\
$Cov(X_{t+h}, X_t) = Cov(a + bZ_0, a + bZ_0) = Var(a + bZ_0) = Var(a) + Var(bZ_0) = a^2 + b^2Var(Z_0) = a^2 + b^2 \sigma^2,$ което не зависи от $t$ за всяко $h$. Следователно процесът е стационарен. 
\end{flushleft}

\textbf{2. $X_t = a + bZ_t + cZ_{t-2}$}
\begin{flushleft}
$EX_t = E(a + bZ_t + cZ_{t-2}) = E(a) + E(bZ_t) + E(cZ_{t-2}) = a + bE(Z_t) + cE(Z_{t-2}) = a,$ което не зависи от $t$. \\

$E{X_t}^2 = E((a + bZ_t + cZ_{t-2})^2) = E(a^2 + b^2{Z_t}^2 + c^2{Z_{t-2}^2} + 2abZ_t + 2acZ_{t-2} + 2bcZ_tZ_{t-2}) = a^2 + b^2\sigma^2 + c^2\sigma^2 = a^2 + (b^2 + c^2)\sigma^2 < \infty$\\

$Cov(X_{t+h}, X_t) = $ \\ 
$Cov(a + bZ_{t+h} + cZ_{t+h-2}, a + bZ_t + cZ_{t-2}) = $ \\
$Cov(bZ_{t+h}, bZ_{t}) +Cov(bZ_{t+h}, cZ_{t-2}) + Cov(cZ_{t+h-2}, bZ_t) + Cov(cZ_{t+h-2}, cZ_{t-2}) =$ \\ 
$b^2Cov(Z_{t+h}, Z_t) + bcCov(Z_{t+h}, Z_{t-2}) + bcCov(Z_{t+h-2}, Z_t) + c^2Cov(Z_{t+h-2}, Z_{t-2}) =$ \\
$b^2E(Z_{t+h}Z_t) + bcE(Z_{t+h}Z_{t-2}) + bcE(Z_{t+h-2}Z_t) + c^2E(Z_{t+h-2}Z_{t-2})$ \\

Получаваме следните случаи: \\
1. $h = 0$: \\
$Cov(X_{t+0}, X_t) = b^2E(Z_{t}Z_t) + bcE(Z_{t}Z_{t-2}) + bcE(Z_{t-2}Z_t) + c^2E(Z_{t-2}Z_{t-2}) = (b^2 + c^2)\sigma^2$

2. $h = 2$: \\
$Cov(X_{t+2}, X_t) = b^2E(Z_{t+2}Z_t) + bcE(Z_{t+2}Z_{t-2}) + bcE(Z_{t}Z_t) + c^2E(Z_{t}Z_{t-2}) = bc\sigma^2$

3. $h = -2$: \\
$Cov(X_{t-2}, X_t) = b^2E(Z_{t-2}Z_t) + bcE(Z_{t-2}Z_{t-2}) + bcE(Z_{t-4}Z_t) + c^2E(Z_{t-4}Z_{t-2}) = bc\sigma^2$

4. $|h| > 2$: \\
$Cov(X_{t+2}, X_t) = 0$ \\
Следователно $Cov(X_{t+0}, X_t)$ не зависи от $t$ за всяко $h$. Следователно процесът е стационарен.

\end{flushleft}

\section*{Задача 5}
\begin{flushleft}

\textbf{1. $X_t = a + bZ_0$}
\begin{flushleft}
\end{flushleft}
\end{flushleft}


\end{flushleft}


\end{document}
